\documentclass{book}
\usepackage[brazilian]{babel}
\usepackage[utf8]{inputenc}

\title{O pacífico dia de amanhã}
\begin{document}
\chapter*{Prefácio}
O livro que você está prestes a ler foi escrito por um ciêntista da computação
não particularmente bem letrado. Ao escrever este livro eu me sinto como um
adolescente com uma idéia para um programa novo, mas que pouco sabe sobre
linguagens de programção. Seu código pode não ser elegante, seu programa pode
ter alguns \emph{bugs} aqui e ali, mas mesmo assim, seu programa final pode ser
útil. No meu caso eu espero que a estória que contarei será interessante,
independente de meus erros gramaticais, de concordancia e da minha pouca
fluência com a lingua (quando comparado a bons escritores).

Eu escrevi o meu melhor e fiz o máximo para manter uma escrita interessante e
bem feita. Noentanto, por comentários feitos por diversos professores de
português ao longo dos anos, sei das minhas limitações. Se você irá ler este
livro mesmo assim, então lhe peço por sugestões. Não serei orgulhoso o
suficiente para não modificar uma frase ou um paragrafo de forma a deixá-lo mais
bonito. Estou aberto, inclusive, para modificações na estória.

Da mesma forma que meus softwares são abertos para quem quiser enviar um
\emph{patch}, este livro é aberto para quem quiser melhorá-lo (enviar um
\emph{patch}). Tenha em mente que nem todo \emph{patch} é aceito aqui ou no
desenvolvimento de softwares. Não fique triste por isso, se você acha que é
relevante o suficiente, por favor, faça um \emph{fork} do livro.

\chapter{Mundo revelado}
%No ano de 2048 foi descoberta uma droga baseada na oxitocina capaz de fazer as
%pessoas confiarem umas nas outras. Todos devem utilizar a droga pela manha,
%dessa forma a sociedade funciona em perfeita harmonia e cada pessoa confia
%totalmente na proxima.
Um calmo som toca em um quarto de tamanho médio. Em uma cama de casal, Rodrigo
rola de um lado para o outro, abrindo o olho com a dificuldade de quem é
repentinamente tirado do fantastico mundo dos sonhos e trazido para o massante
cotidiano. O quarto é predominantemente de cor branca, uma bancada branca colada
à parede se extende por duas das quatro paredes do quarto.  Existe um armário
embutido aos pés da cama. Em cima da bancada existe um copo com água até a
metade, uma caneta e um caderno com alguns desenhos. Nada além
disso é visível.

Rodrigo abre os olhos, suspira fundo e bate palmas duas vezes. A música para de
tocar. Imagens de um jornal televisivo são exibidas como se projetadas no ar. A
ilusão de óptica é causada pela forma com a parede projeta as imagens no vidro
que a recobre. O jovem de 23 anos levanta com preguiça de sua cama, encara o
jornal por alguns instantes -- a reportagem é sobre uma nova alta na bolsa de
valores. De forma indiferente ele tira sua atenção da televisão e olha a
bancada.

Existem várias telas similares a da TV na bancada. Algumas apresentam
\emph{menus}, enquanto outras apresentam textos, previsão do tempo e tudo
mais que o rapaz tenha configurado para aparecer quando ele acorda. Em um dos
menus ele escolhe uma foto de um café, pressionando a região com os dedos. Ele
também seleciona a foto de torradas. Em alguns minutos uma luz na parede pisca.
O jovem abre uma porta na parede e pega suas torradas e café. Ele se senta em
uma cadeira em frente à bancada e come seu café da manhã enquanto lê tirinhas em
sua bancada.

A cozinha de seu apartamento é automatizada, numa sala ao lado do quarto braços
mecânicos criam pratos a partir dos ingredientes disponíveis. As receitas podem
ser requisitadas a partir de bancos de dados disponibilizados por empresas
especializadas. Como as cozinhas são padronizadas os \emph{chefs} são capazes de
escrever algoritmos precisos de como fazer cada prato. A entrega dos
ingredientes pode ser requisitadas em um supermercado. Tudo que o usuário
precisa fazer é informar o que ele deseja comer nos próximos dias.

Junto com o café com torradas Rodrigo recebeu um pequeno copo de papel com duas
pílulas. No final de seu café ele resolveu tomá-las. Suspirou mais uma vez e se
levantou da cadeira. Desta vez ele foi em direção ao armário, onde escolheu
camisa e calças.

Uma das portas do armário era, na verdade, a porta para um pequeno banheiro.
Uma pia com deteção de movimento é a primeira coisa que se pode ver ao entrar no
banheiro. Ao lado da pia a privada e o chuveiro compartilham o mesmo espaço.
Para tomar banho a privada pode ser retraída para dentro da parede.
\end{document}
